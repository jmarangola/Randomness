\documentclass[12pt]{article}
\usepackage[utf8]{inputenc}
\usepackage{amsmath}
\usepackage{amssymb}
\usepackage{amsthm}
% These three packages are from the American Mathematical Society and includes all of the important symbols and operations 
\usepackage{fullpage}
\usepackage{lipsum}
\usepackage{xfrac}

\usepackage{titlesec}
\usepackage{tikz}
\usepackage{mathtools}
\usepackage{siunitx}
\usepackage{changepage}
\usepackage{cancel}
\usepackage{tikz}
\usetikzlibrary{arrows}
\tikzset{
  treenode/.style = {align=center, inner sep=0pt, text centered,
    font=\sffamily},
  arn_n/.style = {treenode, circle, white, font=\sffamily\bfseries, draw=black,
    fill=black, text width=1.5em},% arbre rouge noir, noeud noir
  arn_r/.style = {treenode, circle, red, draw=red, 
    text width=1.5em, very thick},% arbre rouge noir, noeud rouge
  arn_x/.style = {treenode, circle, draw=black,
    minimum width=1.5em, minimum height=0.5em}% arbre rouge noir, nil
}


\sisetup{output-exponent-marker=\ensuremath{\mathrm{e}}}
% By default, an article has some vary large margins to fit the smaller page format.  This allows us to use more standard margins.

%\setlength{\parskip}{1em}
\setlength{\abovedisplayskip}{3pt}
\setlength{\belowdisplayskip}{3pt}
\renewcommand{\vec}[1]{\mathbf{#1}}
\linespread{1.75}

% This gives us a full line break when we write a new paragraph
\titlespacing\section{0pt}{12pt plus 4pt minus 2pt}{0pt plus 2pt minus 2pt}
\titlespacing\subsection{0pt}{12pt plus 4pt minus 2pt}{0pt plus 2pt minus 2pt}
\titlespacing\subsubsection{0pt}{12pt plus 4pt minus 2pt}{0pt plus 2pt minus 2pt}
\title{Randomness and Computation: Written Problem Set 1}
\author{John Marangola}
\date{February 26, 2021}
\begin{document}
\maketitle

%CONSTANTS USED FOR PHYSICS COMPUTATIONS MADE BY ME TO MAKE LIFE EASIER!
% constant term for coulumb's law and electric field 1/4pi epsilon<0>:
\newcommand{\ec}{\frac{1}{4\pi\epsilon_0}}
%for problem 13:
\newcommand{\qa}{\frac{q^2}{a^2}}
% Standard basis vectors: ---------------->
\newcommand{\ih}{\hat{i}}
\newcommand{\jh}{\hat{j}}
\newcommand{\kh}{\hat{k}}
%--------------------------------------------------
% Differentiation at a point:
\newcommand{\at}[2][]{#1|_{#2}}
\section*{1.}
    \begin{enumerate}
        \item[(a)]
        \begin{equation}
            b(n, p, k) = \binom{n}{k}p^{k}\left(1-p\right)^{n-k}
        \end{equation}
        \begin{flushleft}
            Assuming Latouch has the ability that she claims, the probability that Tawna wins the bet $P(X)$ is given by:
        \end{flushleft}
        \begin{align*}
            P(X) &= \sum_{k=7}^{10}{b(10, 0.75, k)} = \sum_{k=7}^{10}{\binom{10}{k}(0.75)^{k}\left(0.25\right)^{10-k}} \\
            &= \binom{10}{7}(0.75)^7 (0.25)^3 +\text{ ... } + \binom{10}{10}(0.75)^{10} \\
            &= 0.7758750916 \approx 0.776
        \end{align*}
        \item[(b)] 
            \begin{align*}
                \Omega = \{1, 2, 3, 4, 5, 6, 7\} 
            \end{align*}
        
    \end{enumerate}
\end{document}
